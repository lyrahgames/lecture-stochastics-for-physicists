% documentclass: article used for scientific journals, short reports, program documentation, etc
% options: fontsize 11, generate document for double sided printing, a4-paper
\documentclass[10pt, twoside, a4paper]{article}

% package for changing page layout
\usepackage{geometry}
\geometry{a4paper, lmargin=40mm, rmargin=45mm, tmargin=40mm, bmargin=45mm}
% set indentation
\setlength{\parindent}{1em}
% set factor for line spacing
% \linespread{1.0}\selectfont
% set (dynamic) additional line spacing
% \setlength{\parskip}{1ex plus 0.5ex minus 0.3ex}

% rigorous formatting (not too much hyphens)
% \fussy
% \sloppy

% package for changing page layout (used to indent whole paragraphs with adjustwidth)
\usepackage{changepage}

% input encoding for special characters (e.g. ä,ü,ö,ß), only for non english text
% options: utf8 as encoding standard, latin1
\usepackage[utf8]{inputenc}
% package for font encoding
\usepackage[T1]{fontenc}
% package for changing used language (especially for more than one language)
% options: ngerman (new spelling) or default: english
\usepackage[ngerman]{babel}
% package for times font
% \usepackage{times}
% package for latin modern fonts
\usepackage{lmodern}

% package for math symbols, functions and environments from ams(american mathematical society)
\usepackage{amsmath}
\usepackage{mathtools}
% package for extended symbols from ams
\usepackage{amssymb}
% package for math black board symbols (e.g. R,Q,Z,...)
\usepackage{bbm}
\usepackage{mathrsfs}
% package for extended symbols from stmaryrd(st mary road)
\usepackage{stmaryrd}

% package for including extern graphics plus scaling and rotating
\usepackage{graphicx}
%package for positioning figures
\usepackage{float}
% package for changing color of font and paper
% options: using names of default colors (e.g red, black)
% \usepackage[usenames]{color}
\usepackage[dvipsnames]{xcolor}
\definecolor{shadecolor}{gray}{0.9}
% package for customising captions
\usepackage[footnotesize, hang]{caption}
% package for customising enumerations (e.g. axioms)
\usepackage{enumitem}
% calc package reimplements \setcounter, \addtocounter, \setlength and \addtolength: commands now accept an infix notation expression
\usepackage{calc}
% package for creating framed, shaded, or differently highlighted regions that can break across pages; environments: framed, oframed, shaded, shaded*, snugshade, snugshade*, leftbar, titled-frame
\usepackage{framed}
% package for creating custom "list of"
% options: titles: do not intefere with standard headings for "list of"
\usepackage[titles]{tocloft}


% change enumeration style of equations
% \renewcommand\theequation{\thesection.\arabic{equation}}

% init list of math for definitions and theorems
\newcommand{\listofmathcall}{Verzeichnis der Definitionen und Sätze}
\newlistof{math}{mathlist}{\listofmathcall}
% add parentheses around argument
\newcommand{\parent}[1]{ \ifx&#1&\else (#1) \fi }
% unnumerated mathematical definition environment definiton
\newenvironment{mathdef*}[2]{
	\begin{framed}
	\noindent
	{ \fontfamily{ppl}\selectfont \textbf{\textsc{#1:}} } ~ #2 
	\par \hfill\\ 
	\fontfamily{lmr}\selectfont \itshape
}{
	\end{framed}
}
% definitions for numerated mathematical definition environment
\newcounter{mathdefc}[section]
\newcommand*{\mathdefnum}{\thesection.\arabic{mathdefc}}
\renewcommand{\themathdefc}{\mathdefnum}
\newenvironment{mathdef}[2]{
	\refstepcounter{mathdefc}
	\addcontentsline{mathlist}{figure}{\protect{\numberline{\mathdefnum}#1 ~ #2}}
	\begin{mathdef*}{#1 \mathdefnum}{#2}
}{
	\end{mathdef*}
}
% standard mathdef calls
\newcommand{\definitioncall}{Definition}
\newenvironment{stddef*}[1][]{ \begin{mathdef*}{\definitioncall}{\parent{#1}} }{ \end{mathdef*} }
\newenvironment{stddef}[1][]{ \begin{mathdef}{\definitioncall}{\parent{#1}} }{ \end{mathdef} }
% unnumerated theorem environment definition
\newenvironment{maththeorem*}[2]{
	\begin{leftbar}
	\noindent
	{ \fontfamily{ppl}\selectfont \textbf{\textsc{#1:}} } ~ #2
	\par \hfill\\ 
	\fontfamily{lmr} \fontshape{it} \selectfont
}{ 
	\end{leftbar}
}
% definitions for numerated theorem environment
\newcounter{maththeoremc}[section]
\newcommand*\maththeoremnum{\thesection.\arabic{maththeoremc}}
\renewcommand{\themaththeoremc}{\maththeoremnum}
\newenvironment{maththeorem}[2]{
	\refstepcounter{maththeoremc}
	\addcontentsline{mathlist}{figure}{\protect{\qquad\numberline{\maththeoremnum}#1 ~ #2}}
	\begin{maththeorem*}{#1 \maththeoremnum}{#2}
}{
	\end{maththeorem*}
}
% standard maththeorem calls
\newcommand{\theoremcall}{Theorem}
\newenvironment{theorem*}[1][]{ \begin{maththeorem*}{\theoremcall}{\parent{#1}} }{ \end{maththeorem*} }
\newenvironment{theorem}[1][]{ \begin{maththeorem}{\theoremcall}{\parent{#1}} }{ \end{maththeorem} }
\newcommand{\lemmacall}{Lemma}
\newenvironment{lemma*}[1][]{ \begin{maththeorem*}{\lemmacall}{\parent{#1}} }{ \end{maththeorem*} }
\newenvironment{lemma}[1][]{ \begin{maththeorem}{\lemmacall}{\parent{#1}} }{ \end{maththeorem} }
\newcommand{\propositioncall}{Proposition}
\newenvironment{proposition*}[1][]{ \begin{maththeorem*}{\propositioncall}{\parent{#1}} }{ \end{maththeorem*} }
\newenvironment{proposition}[1][]{ \begin{maththeorem}{\propositioncall}{\parent{#1}} }{ \end{maththeorem} }
\newcommand{\corollarycall}{Korollar}
\newenvironment{corollary*}[1][]{ \begin{maththeorem*}{\corollarycall}{\parent{#1}} }{ \end{maththeorem*} }
\newenvironment{corollary}[1][]{ \begin{maththeorem}{\corollarycall}{\parent{#1}} }{ \end{maththeorem} }
% q.e.d. definition
\newcommand{\qed}{ \par \hfill \fontfamily{lmr} \fontshape{it} \selectfont \mbox{q.e.d.} \\}
\newcommand{\qedbox}{ \par \hfill $\Box$ \\ }
% proof environment definition for theorems
\newenvironment{mathproof}[1]{
	\par\hfill\\
	\noindent
	{ \fontfamily{lmr}\selectfont \textsc{#1:}}
	\normalfont
	\small
	% \footnotesize
	\begin{adjustwidth}{1em}{}
	\medskip
}{ 
	\end{adjustwidth} 
	% \qed
	\qedbox
}
% standard mathproof calls
\newcommand{\proofcall}{Beweis}
\newenvironment{proof}{ \begin{mathproof}{\proofcall} }{ \end{mathproof} }
\newcommand{\proofideacall}{Beweisidee}
\newenvironment{proofidea}{ \begin{mathproof}{\proofideacall} }{ \end{mathproof} }

% new displaymath command, so that equations will not be stretched
\newcommand{\D}[1]{\mbox{$ #1 $}}
% make unnumerated equation
\newcommand{\E}[1]{\[ #1 \]}
% command for curly brackets
\newcommand{\curlb}[1]{\left\{ #1 \right\}}
% command for box brackets
\newcommand{\boxb}[1]{\left[ #1 \right]}
% command for parentheses/curved brackets
\newcommand{\curvb}[1]{\left( #1 \right)}
% command for angle brackets
\newcommand{\angleb}[1]{\left\langle #1 \right\rangle}
% command for floor brackets
\newcommand{\floorb}[1]{\left\lfloor #1 \right\rfloor}
% command for ceil brackets
\newcommand{\ceilb}[1]{\left\lceil #1 \right\rceil}
% command for creating sets
\newcommand{\set}[2]{ \left\{ #1 \enspace \middle\vert \enspace #2 \right\} }
% command for absolute value
\newcommand{\abs}[1]{\left\vert #1 \right\vert}
\newcommand{\norm}[1]{\left\| #1 \right\|}
% commands for writing limits
\newcommand{\limit}[3]{\, \longrightarrow \, #1, \ #2 \longrightarrow #3}
\newcommand{\Limit}[2]{\lim_{#1 \rightarrow #2}}
% command for differential
\newcommand{\diff}{\mathrm{d}}
\newcommand{\Diff}{\mathrm{D}}
% command for derivative
\newcommand{\Deriv}[3][]{\Diff_{#2}^{#1}#3}
\newcommand{\deriv}[3][]{\dfrac{\diff^{#1}#2(#3)}{\diff #3^{#1}}}
% command for integral
\newcommand{\integral}[4]{\int_{#1}^{#2} #3\ \diff #4}
\newcommand{\Integral}[4]{\int\limits_{#1}^{#2} #3\ \diff #4}
% mathematical definitions (standard sets)
\newcommand{\SC}{\mathbbm{C}}
\newcommand{\SR}{\mathbbm{R}}
% \newcommand{\SRP}{\SR^+}
% \newcommand{\SRPN}{\SRP_0}
\newcommand{\SN}{\mathbbm{N}}
% \newcommand{\SNN}{\SN_0}
\newcommand{\SZ}{\mathbbm{Z}}
\newcommand{\SQ}{\mathbbm{Q}}
% \newcommand{\SQP}{\SQ^+}
% \newcommand{\SQPN}{\SQP_0}
\newcommand{\SP}{\mathcal{P}}

\DeclareMathOperator{\real}{Re} % real part
\DeclareMathOperator{\imag}{Im} % imaginary part
\newcommand{\FF}{\mathcal{F}} % fourier transform
\newcommand{\FE}{\mathbb{E}} % expectation
\DeclareMathOperator{\var}{var} % variance
\newcommand{\FN}{\mathcal{N}} % normal distribution

% command for physical units
\newcommand{\unit}[1]{\, \text{#1}}


% package for init listings(non-formatted  text) e.g. different source codes
\usepackage{listings}


% definitions for listing colors
\definecolor{codeDarkGray}{gray}{0.2}
\definecolor{codeGray}{gray}{0.4}
\definecolor{codeLightGray}{gray}{0.9}
% predefinitions for listings
\newcommand{\listingcall}{Listing}
\newlength{\listingframemargin}
\setlength{\listingframemargin}{1em}
\newlength{\listingmargin}
\setlength{\listingmargin}{0.1\textwidth}
% \newlength{\listingwidth}
% \setlength{\listingwidth}{ ( \textwidth - \listingmargin * \real{2} + \listingframemargin * \real{2} ) }
% definitions for list of listings
\newcommand{\listoflistingscall}{\listingcall -Verzeichnis}
\newlistof{listings}{listinglist}{\listoflistingscall}
% style definition for standard code listings
\lstdefinestyle{std}{
	belowcaptionskip=0.5\baselineskip,
	breaklines=true,
	frameround=false,
	frame=tb,
	xleftmargin=0em,
	xrightmargin=0em,
	showstringspaces=false,
	showtabs=false,
	% tab=\smash{\rule[-.2\baselineskip]{.4pt}{\baselineskip}\kern.5em},
	basicstyle= \fontfamily{pcr}\selectfont\footnotesize\bfseries,
	keywordstyle= \bfseries\color{MidnightBlue}, %\color{codeDarkGray},
	commentstyle= \itshape\color{codeGray},
	identifierstyle=\color{codeDarkGray},
	stringstyle=\color{BurntOrange}, %\color{codeDarkGray},
	numberstyle=\tiny\ttfamily,
	% numbers=left,
	numbersep = 2em,
	% numberstep = 5,
	% captionpos=t,
	tabsize=4,
	backgroundcolor=\color{codeLightGray},
	framexleftmargin=\listingframemargin,
	framexrightmargin=\listingframemargin
}
% definition for unnumerated listing
\newcommand{\inputlistingn}[3][]{
	\begin{center}
		\begin{adjustwidth}{\listingmargin}{\listingmargin}
			\centerline{ {\fontfamily{lmr}\selectfont\scshape \listingcall:}\quad #2 }
			\lstinputlisting[style=std, #1]{#3}
		\end{adjustwidth}
	\end{center}
}
% definition for numerated listing
\newcounter{listingc}[section]
\newcommand*\listingnum{\thesection.\arabic{listingc}}
\renewcommand{\thelistingc}{\listingnum}
\newcommand{\inputlisting}[3][]{
	\refstepcounter{listingc}
	\addcontentsline{listinglist}{figure}{\protect{\numberline{\listingnum:} #2 } }
	\inputlistingn[#1]{#2}{#3}
}


% package for including csv-tables from file
% \usepackage{csvsimple}
% package for creating, loading and manipulating databases
\usepackage{datatool}

% package for converting eps-files to pdf-files and then include them
\usepackage{epstopdf}
% use another program (ps2pdf) for converting
% !!! important: set shell_escape=1 in /etc/texmf/texmf.cnf (Linux/Ubuntu 12.04) for allowing to use other programs
% !!!			or use the command line with -shell-escape
\epstopdfDeclareGraphicsRule{.eps}{pdf}{.pdf}{
ps2pdf -dEPSCrop #1 \OutputFile
}


% package for reference to last page (output number of last page)
\usepackage{lastpage}
% package for using header and footer
% options: automate terms of right and left marks
\usepackage[automark]{scrpage2}
% \setlength{\headheight}{4\baselineskip}
% set style for footer and header
\pagestyle{scrheadings}
% \pagestyle{headings}
% clear pagestyle for redefining
\clearscrheadfoot
% set header and footer: use <xx>head/foot[]{Text} (i...inner, o...outer, c...center, o...odd, e...even, l...left, r...right)
\ihead[]{\leftmark}
\ohead[]{\rightmark}
\cfoot[]{\newline\newline\newline\pagemark}
% use that for mark to last page: \pageref{LastPage}
% set header separation line
\setheadsepline[\textwidth]{0.5pt}
% set foot separation line
\setfootsepline[\textwidth]{0.5pt}


\title{Stochastik II für Physiker}
\author{Markus Pawellek}
% \date{}		



% begin the document
\begin{document}

	\paragraph{Bemerkung:}
Wesentlich schwieriger ist das Testen von Hypothesen über den Verteilungstyp, ohne den Wert der Parameter zu spezifizieren.
Zum Beispiel:
\begin{alignat*}{3}
	H_0: P_{X_1} \in \set{\mathcal{N}_{\mu,\sigma}}{\mu\in\SR, \sigma^2>0} \\
	H_1: P_{X_1} \notin \set{\mathcal{N}_{\mu,\sigma}}{\mu\in\SR, \sigma^2>0}
\end{alignat*}
In diesem Falle sei die Testgröße
\[
	T(X_1,\ldots,X_n) = \sqrt{n} \sup_{t\in\SR} \abs{ \hat{F}(t) - F_{\hat{\mu},\hat{\sigma}^2}(t) }
\]
Wobei hier $F_{\hat{\mu},\hat{\sigma}^2}$ die Verteilungsfunktion zu $\mathcal{N}_{\hat{\mu},\hat{\sigma}}$ ist.
Weiterhin lassen sich Verteilungsfunktionen auch anhand von Simulationen bestimmen.
\begin{itemize}
	\item Kolmojorow-Smiruov-Test wird als extrem konservativ bezeichnet und verwirft zu selten $H_0$
	\item grafische Methode für Anpassungstest auf Normalverteilung
\end{itemize}

\subsection{Zweistichproben-Test} % (fold)
\label{sub:zweistichproben_test}

	Gegeben seien zwei konkrete Stichproben vom Umfang $n\in\SN$ beziehungsweise $m\in\SN$ aus zwei Grundgesamtheiten
	\begin{align*}
		x_1,\ldots,x_n\\
		y_1,\ldots,y_m
	\end{align*}
	Dies tritt zum Beispiel beim Vergleich zwei verschiedener Produktionsverfahren auf.
	Für die zugehörige mathematische Stichprobe gilt dann
	\begin{align*}
		X_1,\ldots,X_n \textit{ i.i.d gemäß } P_{\theta_1} \\
		Y_1,\ldots,Y_m \textit{ i.i.d gemäß } P_{\theta_2}
	\end{align*}
	Das Ziel ist nun der Vergleich von Merkmalen der Verteilungstypen wie zum Beispiel
	\begin{itemize}
		\item Erwartungswert
		\item Varianz
		\item verschiedene Quantile
		\item Verteilungsfunktion 
	\end{itemize}
	Für uns soll es hier reichen die Erwartungswerte zweier normalverteilter Grundgesamtheiten mit unbekannter Varianz zu betrachten.
	Dies wird mithilfe eines Zweistichproben-$t$-Test ermöglicht.
	Es sei also folgender statistischer Raum gegeben
	\[
		\curvb{ \SR^n\times\SR^m, \mathcal{R}_{n+m}, \set{\mathcal{N}_{\mu_1, \sigma^2} \otimes \mathcal{N}_{\mu_2, \sigma^2} }{ \mu_1,\mu_2\in\SR, \sigma^2>0 } }
	\]
	Es ergeben sich dann als Beispiel folgende einseitige Alternativhypothesen
	\begin{align*}
		H_0: \mu_1\leq\mu_2 \\
		H_1: \mu_1>\mu_2
	\end{align*}
	Es soll dabei folgende Testgröße verwendet werden
	\[
		T(x_1,\ldots,x_n,y_1,\ldots,y_m) = \sqrt{\frac{nm}{n+m}}\cdot\frac{\bar{x}-\bar{y}}{\sqrt{\widehat{\hat{\sigma}^2}}}
	\]
	mit
	\begin{align*}
		\widehat{\hat{\sigma}^2} &= \frac{1}{m+n-2}\boxb{ \sum_{i=1}^n \curvb{x_i-\bar{x}}^2 + \sum_{i=1}^m \curvb{y_i-\bar{y}}^2 } \\
		&= \frac{1}{m+n-2}\boxb{ (n-1)\hat{\sigma}_x^2 + (m-1)\hat{\sigma}_y^2 }
	\end{align*}
	Es folgt für den Fall $ \mu_1=\mu_2 $ durch Rechnung
	\[
		T(X_1,\ldots,X_n,Y_1,\dots,Y_m) \sim t_{n+m-2}
	\]
	Der kritische Bereich kann also analog zu den vorherigen Hypothesentests gewählt werden.
	\[
		K = [t_{n+m-2,1-\alpha},\infty)
	\]
	Sollte es doch passieren, dass $\mathrm{var}(X_1)\neq\mathrm{var}(Y_1)$, so empfiehlt sich die Anwendung des Welch-Tests (hier nicht beschrieben).
	Ein Test zum Vergleich der Varianzen für normalverteilte Grundgesamtheiten ist der sogenannte $F$-Test (Erwartungswerte müssen nicht gleich sein).


	\section{statistische Methoden für zweidimensionale Stichproben} % (fold)
	\label{sec:statistische_methoden_f_r_zweidimensionale_stichproben}
	\footnote{Multivariatstatistik}

		Man betrachtet nun sogenannte gepaarte Stichproben
		\[
			(x_1,y_1),\ldots,(x_n,y_n) \in \SR^{2n}
		\]
		Es geht also inhaltlich um Paare von Messwerten zu einem bestimmten Zeitpunkt\footnote{oder auch an einem Objekt}.
		Verschiedene Fragestellungen sind nun zum Beispiel
		\begin{description}
			\item[Korrelationsanalyse:] Sind $X_1$ und $Y_1$ unabhängig?
			\item[Regressionsanalyse:] Wie sieht die funktionale Abhängigkeit von $X_1$ und $Y_1$ aus, wenn sie nicht unabhängig sind?
			\item[Varianzanalyse:] Untersuchung von Parametervektoren.
			\item[Diskriminanzanalyse:] Trennverfahren zur Zuordnung von Gruppen
			\item[Faktorenanalyse:] Untersuchung von Kovarianzmatrizen
			\item[Clusteranalyse:] Klassifikation von Werten
		\end{description}
		Dennoch werden wir auch hier grundsätzlich Punktschätzungen, Konfidenzschätzungen und Tests betrachten.
		
		\subsection{Regressionsanalyse} % (fold)
		\label{sub:regressionsanalyse}
		
			Sei $(x_1,y_1),\ldots,(x_n,y_n)$ für $n\in\SN$ eine konkrete Stichprobe.
			Das Modell der zugehörigen mathematischen Stichprobe sei nun
			\[ (x_1,Y_1),\ldots,(x_n,Y_n) \]
			wobei $Y_1,\ldots,Y_n$ Zufallsvariablen sind.
			Die Wahl des Regressionsansatzes sei intuitiv für alle $i=1,\ldots,n$ gegeben.
			\[ y_i = \beta_1 + \beta_2 x_i + u_i \]
			Damit folgt auch das stochastische Modell.
			Es ist ein einfaches lineares Regressionsmodell.
			\[ Y_i = \beta_1 + \beta_2 x_i + U_i \quad \text{für alle } i=1,\ldots,n \]
			Die $U_1,\ldots,U_n$ sind dabei i.i.d. Zufallsvariablen mit
			\[ \FE U_i = 0,\qquad \var U_i = \sigma^2 >0 \]

			\textbf{Bemerkung:}
			\begin{itemize}
				\item Die $u_i$ können nicht beobachtet werden.
				\item Für die unbekannten Parameter $\beta_1,\beta_2,\sigma^2$ verwendet man Folgendes
				\begin{itemize}
					\item Punkt- und Konfidenzbereichsschätzungen
					\item Testen von Hypothesen (zum Beispiel über $(\beta_1, \beta_2)$)
					\item Punkt- und Konfidenzschätzungen für Prognosewerte $Y(x)$ an einer Stelle $x\in\SR$
					\item Test auf Adäquatheit des Modells (zum Beispiel goodness-of-fit-Test oder lack-of-fit-Test)
				\end{itemize}
			\end{itemize}

			Es sei nun
			\[ \underline{\beta} := \begin{pmatrix}
				\beta_1 \\
				\beta_2
			\end{pmatrix} \]
			Ziel ist es nun eine Punktschätzung für $\underline{\beta}$ durch die Methode der kleinsten Quadrate zu erhalten.
			Man wählt $\hat{\underline{\beta}}\in\SR^2$ als Lösung der Optimierungsaufgabe
			\[ S(\hat{\underline{\beta}}) = \sum_{i=1}^n \boxb{y_i - \curvb{\beta_1 + \beta_2 x_i}}^2 \stackrel{!}{=} \min_{\underline{\beta}\in\SR^2} S(\underline{\beta}) \]

		% subsection regressionsanalyse (end)

	% section statistische_methoden_f_r_zweidimensionale_stichproben (end)

% subsection zweistichproben_test (end)

\end{document}

